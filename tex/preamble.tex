\usepackage{amsmath}
\usepackage{amsthm}
\usepackage{nameref}
\usepackage{url}
\usepackage[backend=biber, natbib=true, style=numeric]{biblatex}
\usepackage{amsfonts}
\usepackage{bm}
\usepackage{graphicx}
\usepackage{subcaption}
\usepackage{accents}
\usepackage{mathtools}
\usepackage[utf8]{inputenc}
\usepackage{booktabs}
\usepackage{units}

\usepackage[ruled,vlined]{algorithm2e}
\usepackage{algpseudocode}

\usepackage{xargs}
\usepackage{xspace}
\usepackage{colortbl}

% Inserts a blank page
\newcommand{\blankpage}{\newpage\hbox{}\thispagestyle{empty}\newpage}

% Citation as sidenote
\renewcommandx{\cite}[3][1={0pt},2={}]{\sidenote[][#1]{\fullcite[#2]{#3}}}

% Generates the index
\usepackage{makeidx}
\makeindex

% Set default
\setkeys{Gin}{width=\linewidth,totalheight=\textheight,keepaspectratio}

\usepackage{fancyvrb}
\fvset{fontsize=\normalsize}

%%%% Kevin Godny's code for title page and contents from https://groups.google.com/forum/#!topic/tufte-latex/ujdzrktC1BQ
\makeatletter
\renewcommand{\maketitlepage}{%
\begingroup%
\setlength{\parindent}{0pt}

{\fontsize{24}{24}\selectfont\textit{\@author}\par}

\vspace{1.75in}{\fontsize{36}{54}\selectfont\@title\par}

\vspace{0.5in}{\fontsize{14}{14}\selectfont\textsf{\smallcaps{\@date}}\par}

\vfill{\fontsize{14}{14}\selectfont\textit{\@publisher}\par}

\thispagestyle{empty}
\endgroup
}
\makeatother

\titlecontents{part}%
    [0pt]% distance from left margin
    {\addvspace{0.5\baselineskip}}% above (global formatting of entry)
    {\allcaps{Part~\thecontentslabel}\allcaps}% before w/ label (label = ``Part I'')
    {\allcaps{Part~\thecontentslabel}\allcaps}% before w/o label
    {}% filler and page (leaders and page num)
    [\vspace*{0.5\baselineskip}]% after

\titlecontents{chapter}%
    [4em]% distance from left margin
    {\addvspace{0.5\baselineskip}}% above (global formatting of entry)
    {\contentslabel{2em}\textit}% before w/ label (label = ``Chapter 1'')
    {\hspace{0em}\textit}% before w/o label
    {\qquad\thecontentspage}% filler and page (leaders and page num)
    [\vspace*{0.25\baselineskip}]% after
    
\titlecontents{section}%
    [7em]% distance from left margin
    {}% above (global formatting of entry)
    {\contentslabel{2em}\textit}% before w/ label (label = ``Chapter 1'')
    {\hspace{0em}\textit}% before w/o label
    {\qquad\thecontentspage}% filler and page (leaders and page num)
    [\vspace*{0.1\baselineskip}]% after
%%%% End additional code by Kevin Godby and Joe Lorenzetti

% Useful for modifying the header to include chapter title
\renewcommand{\chaptermark}[1]{%
\markboth{#1}{}}

% Converts headings
\let\subsubsection\subsection
\let\subsection\section

\newcommand{\notessection}[1]{\section*{#1}}
\setcounter{secnumdepth}{2}


% Changing the overall width of the page a bit, adding more width to text and removing from left margin. Comment this out to return to normal
\geometry{
  left=0.75in, % left margin
  textwidth=30pc, % main text block
  marginparsep=2pc, % gutter between main text block and margin notes
  marginparwidth=12pc % width of margin notes
}



% Default fixed font does not support bold face
\DeclareFixedFont{\ttb}{T1}{txtt}{bx}{n}{9} % for bold
\DeclareFixedFont{\ttm}{T1}{txtt}{m}{n}{9}  % for normal

% Custom colors
\usepackage{color}
\definecolor{deepblue}{rgb}{0,0,0.5}
\definecolor{deepred}{rgb}{0.6,0,0}
\definecolor{deepgreen}{rgb}{0,0.5,0}
\definecolor{backcolour}{rgb}{0.95,0.95,0.95}


% Python and other code listing environment
\usepackage{listings}
\newcommand\pythonstyle{\lstset{
language=Python,
basicstyle=\ttm,
otherkeywords={self},             % Add keywords here
keywordstyle=\ttb\color{deepblue},
emph={MyClass,__init__},          % Custom highlighting
emphstyle=\ttb\color{deepred},    % Custom highlighting style
stringstyle=\color{deepgreen},
frame=tb,                         % Any extra options here
backgroundcolor=\color{backcolour},
showstringspaces=false,            % 
breakatwhitespace=false,
framexleftmargin=1em,
xleftmargin=1em,
}}

\newcommand\pythonstylenoborder{\lstset{
language=Python,
basicstyle=\ttm,
otherkeywords={self},             % Add keywords here
keywordstyle=\ttb\color{deepblue},
emph={MyClass,__init__},          % Custom highlighting
emphstyle=\ttb\color{deepred},    % Custom highlighting style
stringstyle=\color{deepgreen},                       % Any extra options here
backgroundcolor=\color{backcolour},
showstringspaces=false,            % 
breakatwhitespace=false,
framexleftmargin=1em,
xleftmargin=1em,
}}

\newcommand\gencodestyle{\lstset{
basicstyle=\ttm,
backgroundcolor=\color{backcolour},
showstringspaces=false,            % 
breakatwhitespace=false,
framexleftmargin=1em,
xleftmargin=1em,
}}

% Python environment
\lstnewenvironment{python}[1][]
{
\pythonstyle
\lstset{#1}
}
{}

% Python environment with no borders
\lstnewenvironment{pythonnoborder}[1][]
{
\pythonstylenoborder
\lstset{#1}
}
{}

% Python environment
\lstnewenvironment{gencode}[1][]
{
\gencodestyle
\lstset{#1}
}
{}

% Python for external files
\newcommand\pythonexternal[2][]{{
\pythonstyle
\lstinputlisting[#1]{#2}}}

% Python for inline
\newcommand\pythoninline[1]{{\pythonstyle\lstinline!#1!}}

%------------------------------

\theoremstyle{plain}
\newtheorem{theorem}{Theorem}[section]
\newtheorem{definition}[theorem]{Definition}
\theoremstyle{definition}
\newtheorem{example}{Example}[section]

\DeclareMathOperator*{\argmax}{arg\,max}
\DeclareMathOperator*{\argmin}{arg\,min}

\newcommand{\mytilde}{\raise.17ex\hbox{$\scriptstyle\mathtt{\sim}$}} 
\newcommand{\bxi}{\bm{\xi}}
\newcommand{\x}{\bm{x}}
\newcommand{\y}{\bm{y}}
\newcommand{\ac}{\bm{u}}
\newcommand{\z}{\bm{z}}
\newcommand{\p}{\bm{p}}
\newcommand{\blam}{\bm{\lambda}}
\newcommand{\bnu}{\bm{\nu}}
\newcommand{\bu}{\bm{u}}
\newcommand{\q}{\bm{q}}
\newcommand{\bv}{\bm{v}}
\newcommand{\bmu}{\bm{\mu}}
\newcommand{\bSigma}{\bm{\Sigma}}
\newcommand{\m}{\bm{m}}
\newcommand{\w}{\bm{w}}
\newcommand{\bc}{\bm{c}}
\newcommand{\btheta}{\bm{\theta}}
\newcommand{\bphi}{\bm{\phi}}
\newcommand{\btau}{\bm{\tau}}
\newcommand{\f}{\bm{f}}
\newcommand{\bd}{\bm{d}}
\newcommand{\R}{\mathbb{R}}
\newcommand{\C}{\mathcal{C}}
\newcommand{\Space}{\mathcal{S}}
\newcommand{\E}{\mathbb{E}}
\newcommand{\X}{\mathcal{X}}
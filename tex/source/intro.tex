\section*{Forward}
This collection of notes is meant to provide a fundamental understanding of the theoretical and algorithmic aspects associated with robotic autonomy\footnote{The field of robotic autonomy is vast and diverse, encompassing theory and algorithms from many fields of science, technology, and engineering. These notes cannot cover all material and therefore focuses on the most foundational and widely used techniques.}. In particular, these notes cover topics spanning the three main pillars of autonomy: motion planning and control, perception, and decision-making, and also include some information on useful software tools for robot programming, such as the Robot Operating System (ROS). By avoiding extremely in-depth discussions on specific algorithms or techniques, these notes focus on providing a high-level understanding of the full ``autonomy stack'' and are a good starting point for any engineer or researcher interested in robotics. Some other great references that cover a wide range of robotics topics include:

\vspace{\baselineskip}

\fullcite{SiegwartNourbakhshEtAl2011}

\vspace{\baselineskip}

\fullcite{ThrunBurgardEtAl2005}

\vspace{\baselineskip}

While these notes are meant to be as self-contained as is practical, prior knowledge of several topics is generally assumed. Specifically, familiarity with the basics of calculus, differential equations, linear algebra, probability and statistics, and programming is helpful.

\subsection*{Acknowledgments}
These notes accompany (and are based largely on the content of) the courses \textit{AA274A: Principles of Robot Autonomy I} and \textit{AA274B: Principles of Robot Autonomy II}\footnote{Co-taught with Professors Jeannette Bogh and Dorsa Sadigh.} at Stanford University. We would therefore like to acknowledge the students who have taken the course and provided useful feedback since its initial offering in 2017. Special acknowledgements are also reserved for the course assistants:\\
\emph{AA274A, Winter 2017:} Andrew Bylard, Benoit Landry, Ed Schmerling,\\
\emph{AA274A, Winter 2018:} Tommy Hu, Benoit Landry, Karen Leung, Ed Schmerling,\\
\emph{AA274A, Winter 2019:} Christopher Covert, Amine Elhafsi, Karen Leung, Apoorva Sharma,\\
\emph{AA274A, Autumn 2019:} Andrew Bylard, Boris Ivanovic, Jenna Lee,  Toki Migimatsu, Apoorva Sharma,\\
\emph{AA274A, Autumn 2020:} Somrita Banerjee, Abhishek Cauligi, Boris Ivanovic, Mengxi Li, Joseph Lorenzetti,\\
\emph{AA274B, Winter 2020:} Ashar Alam, Erdem B{\i}y{\i}k, Jenna Lee, Toki Migimatsu, \\
\emph{AA274B, Winter 2021:} Erdem B{\i}y{\i}k, Abhishek Cauligi,\\
who were instrumental in developing and refining the course material. In large part, additional material for the course such as homework and lectures are also publicly available\footnote{\url{https://github.com/PrinciplesofRobotAutonomy/CourseMaterials}}.